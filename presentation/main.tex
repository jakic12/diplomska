\documentclass{beamer}

\usepackage[utf8x]{inputenc} 
\usepackage[slovene,english]{babel}
\usepackage{amssymb}
\usepackage{amsmath}
\usepackage{amsthm}
\usepackage{svg}
\usepackage{ulem}
\usepackage{graphicx}

\usetheme{Madrid}

\setbeamercovered{transparent=25} % pause environment is transparent

% command semitransp to make text semitransparent (actually mixes text with background color to make it appear semitransparent)
\newcommand{\semitransp}[2][25]{\color{fg!#1}#2\color{fg}}

\title{Digitalna topologija na grafih}
\subtitle{Predstavitev diplomskega dela}
\author{Jakob Drusany}
\institute[UL FRI, UL FMF]{Fakulteta za računalništvo in informatiko\\Fakulteta za matematiko in fiziko\\Univerza v Ljubljani}
\date{\selectlanguage{slovene}\today}



\begin{document}

\begin{frame}
    \titlepage
    {\centering \small Mentor: prof. dr. Petar Pavešić\par}
\end{frame}

\begin{frame}
\frametitle{Table of Contents}
\tableofcontents
\end{frame}

% Automatically add a table of contents slide at the beginning of each section
% do i need this?
\AtBeginSection[]
{
  \begin{frame}
    \frametitle{Table of Contents}
    \tableofcontents[currentsection]
  \end{frame}
}

\section{Uvod}
\subsection{Motivacija}
\begin{frame}[t]
    \frametitle{Motivacija}
    Zakaj hočemo topologije na slikah?
\end{frame}
\subsection{Teoretične osnove}
\begin{frame}[t]
\frametitle{Teoretične osnove}
\begin{block}{Definicija}
    \alert{Topologija} (ali topološka struktura) na množici $X$ je družina $\mathcal{T}$ podmnožic
    $X$, ki zadošča naslednjim zahtevam:
    \begin{itemize}
        \item[(1)] prazna množica in $X$ sta elementa $\mathcal{T}$;
        \item[(2)] unija poljubne poddružine $\mathcal{T}$ je element $\mathcal{T}$;
        \item[(3)] presek poljubne končne poddružine $\mathcal{T}$ je element $\mathcal{T}$.
    \end{itemize}
\end{block}
\vspace{.25cm}
\pause
Elemente $\mathcal{T}$ imenujemo \alert{odprte množice} v $X$. \alert{Topološki prostor}
($X$, $\mathcal{T}$) je množica $X$, opremljena s topologijo $\mathcal{T}$ \\ \vspace{.25cm}
\alert{Okolica} točke $x \in X$ je vsaka podmnožica $V \subseteq X$, ki vsebuje
odprto množico $U$, ki vsebuje $x$.
\end{frame}

\begin{frame}[t]
\frametitle{Teoretične osnove}
\begin{block}{Definicija}
    \alert{Topologija Aleksandrova} na množici $X$ je družina $\mathcal{T}$ podmnožic
    $X$, ki zadošča naslednjim zahtevam:
    \begin{itemize}
        \item[(1)] prazna množica in $X$ sta elementa $\mathcal{T}$;
        \item[(2)] unija poljubne poddružine $\mathcal{T}$ je element $\mathcal{T}$;
        \item[(3)] presek poljubne \sout{končne} poddružine $\mathcal{T}$ je element $\mathcal{T}$.
    \end{itemize}
\end{block}
\pause
\vspace{.25cm}
Presek vseh odprtih množic, ki vsebujejo točko $x$ je \alert{najmanjša odprta okolica} točke $x$, ki jo označimo z $U_x$.
\end{frame}

\begin{frame}
    \frametitle{TODO?}
    Dodamo lahko še sekcijo o ločitvenih aksiomih
\end{frame}

\section{Končne topologije, delne urejenosti in celični kompleksi}
\begin{frame}[t] % top align
    \frametitle{TODO}
    \vspace{-8mm}
    % NO TRANSPARENCY
    \[
        \text{Delna urejenost}
        \quad
        \raisebox{-3pt}{\scalebox{2}{$\substack{\longleftarrow\\[-1em] \longrightarrow }$}}
        \quad
        \text{Končna topologija}
        \quad
        \raisebox{-3pt}{\scalebox{2}{$\substack{\longleftarrow\\[-1em] \longrightarrow }$}}
        \quad
        \text{Celični kompleks}
    \]
    \vspace{1cm}
    NO TRANSPARENCY
\end{frame}

\begin{frame}[t] % top align
    \frametitle{TODO}
    \vspace{-8mm}
    % FIRST LEFT 
    \[
        \text{Delna urejenost}
        \quad
        \raisebox{-3pt}{\scalebox{2}{$\substack{\longleftarrow\\[-1em] \semitransp{\longrightarrow} }$}}
        \quad
        \text{Končna topologija}
        \quad
        \semitransp{
            \raisebox{-3pt}{\scalebox{2}{$\substack{\longleftarrow\\[-1em] \longrightarrow }$}}
            \quad
            \text{Celični kompleks}
        }
    \]
    \vspace{1cm}
    FIRST LEFT 
\end{frame}
\begin{frame}[t] % top align
    \frametitle{TODO}
    \vspace{-8mm}
    % FIRST RIGHT 
    \[
        \text{Delna urejenost}
        \quad
        \raisebox{-3pt}{\scalebox{2}{$\substack{\semitransp{\longleftarrow}\\[-1em] \longrightarrow }$}}
        \quad
        \text{Končna topologija}
        \quad
        \semitransp{
            \raisebox{-3pt}{\scalebox{2}{$\substack{\longleftarrow\\[-1em] \longrightarrow }$}}
            \quad
            \text{Celični kompleks}
        }
    \]
    \vspace{1cm}
    FIRST RIGHT 
\end{frame}
\begin{frame}[t] % top align
    \frametitle{TODO}
    \vspace{-8mm}
    % SECOND LEFT 
    \[
        \semitransp{
            \text{Delna urejenost}
            \quad
            \raisebox{-3pt}{\scalebox{2}{$\substack{\longleftarrow\\[-1em] \longrightarrow }$}}
            \quad
        }
        \text{Končna topologija}
        \quad
        \raisebox{-3pt}{\scalebox{2}{$\substack{\longleftarrow\\[-1em] \semitransp{\longrightarrow} }$}}
        \quad
        \text{Celični kompleks}
    \]
    \vspace{1cm}
    SECOND LEFT 
\end{frame}
\begin{frame}[t] % top align
    \frametitle{TODO}
    \vspace{-8mm}
    % SECOND RIGHT 
    \[
        \semitransp{
        \text{Delna urejenost}
        \quad
        \raisebox{-3pt}{\scalebox{2}{$\substack{\longleftarrow\\[-1em] \longrightarrow }$}}
        \quad
        }
        \text{Končna topologija}
        \quad
        \raisebox{-3pt}{\scalebox{2}{$\substack{\semitransp{\longleftarrow}\\[-1em] \longrightarrow }$}}
        \quad
        \text{Celični kompleks}
    \]
    \vspace{1cm}
    SECOND RIGHT 
\end{frame}
\subsection{Povezava končnih topologij in delnih urejenosti}
\subsection{Simplicialni kompleksi}
\subsection{Povezava topoloških prostorov in simplicialnih kompleksov}

\section{Digitalni prostori}
\subsection{Topologije na grafih}
\subsection{Kompatibilne topologije na dvodelnih grafih}
\subsection{Obstoj kompatibilne topologije na grafu}
\subsection{Celični kompleksi}

\begin{frame}
\frametitle{Outline}
\end{frame}

\end{document}