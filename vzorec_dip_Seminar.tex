%%%%%%%%%%%%%%%%%%%%%%%%%%%%%%%%%%%%%%%%
% datoteka diploma-vzorec.tex
%
% vzorčna datoteka za pisanje diplomskega dela v formatu LaTeX
% na UL Fakulteti za računalništvo in informatiko
%
% vkup spravil Gašper Fijavž, december 2010
% 
%
%
% verzija 12. februar 2014 (besedilo teme, seznam kratic, popravki Gašper Fijavž)
% verzija 10. marec 2014 (redakcijski popravki Zoran Bosnić)
% verzija 11. marec 2014 (redakcijski popravki Gašper Fijavž)
% verzija 15. april 2014 (pdf/a 1b compliance, not really - just claiming, Damjan Cvetan, Gašper Fijavž)
% verzija 23. april 2014 (privzeto cc licenca)
% verzija 16. september 2014 (odmiki strain od roba)
% verzija 28. oktober 2014 (odstranil vpisno številko)
% verija 5. februar 2015 (Literatura v kazalu, online literatura)
% verzija 25. september 2015 (angl. naslov v izjavi o avtorstvu)
% verzija 26. februar 2016 (UL izjava o avtorstvu)
% verzija 16. april 2016 (odstranjena izjava o avtorstvu)
% verzija 5. junij 2016 (Franc Solina dodal vrstice, ki jih je označil s svojim imenom)


\documentclass[a4paper, 12pt]{book}
%\documentclass[a4paper, 12pt, draft]{book}  Nalogo preverite tudi z opcijo draft, ki vam bo pokazala, katere vrstice so predolge!



\usepackage[utf8x]{inputenc}   % omogoča uporabo slovenskih črk kodiranih v formatu UTF-8
\usepackage[slovene,english]{babel}    % naloži, med drugim, slovenske delilne vzorce
\usepackage[pdftex]{graphicx}  % omogoča vlaganje slik različnih formatov
\usepackage{fancyhdr}          % poskrbi, na primer, za glave strani
\usepackage{amssymb}           % dodatni simboli
\usepackage{amsmath}           % eqref, npr.
%\usepackage{hyperxmp}
\usepackage[hyphens]{url}  % dodal Solina
\usepackage{comment}       % dodal Solina

\usepackage[pdftex, colorlinks=true,
						citecolor=black, filecolor=black, 
						linkcolor=black, urlcolor=black,
						pagebackref=false, 
						pdfproducer={LaTeX}, pdfcreator={LaTeX}, hidelinks]{hyperref}

\usepackage{color}       % dodal Solina
\usepackage{soul}       % dodal Solina
\usepackage[numbers]{natbib}  % dodal Solina

%%%%%%%%%%%%%%%%%%%%%%%%%%%%%%%%%%%%%%%%
%	DIPLOMA INFO
%%%%%%%%%%%%%%%%%%%%%%%%%%%%%%%%%%%%%%%%
\newcommand{\ttitle}{Vzorec diplomskega dela}
\newcommand{\ttitleEn}{Diploma thesis sample}
\newcommand{\tsubject}{\ttitle}
\newcommand{\tsubjectEn}{\ttitleEn}
\newcommand{\tauthor}{Matjaž Kralj}
\newcommand{\tkeywords}{računalnik, računalnik, računalnik}
\newcommand{\tkeywordsEn}{computer, computer, computer}


%%%%%%%%%%%%%%%%%%%%%%%%%%%%%%%%%%%%%%%%
%	HYPERREF SETUP
%%%%%%%%%%%%%%%%%%%%%%%%%%%%%%%%%%%%%%%%
\hypersetup{pdftitle={\ttitle}}
\hypersetup{pdfsubject=\ttitleEn}
\hypersetup{pdfauthor={\tauthor, matjaz.kralj@fri.uni-lj.si}}
\hypersetup{pdfkeywords=\tkeywordsEn}


 


%%%%%%%%%%%%%%%%%%%%%%%%%%%%%%%%%%%%%%%%
% postavitev strani
%%%%%%%%%%%%%%%%%%%%%%%%%%%%%%%%%%%%%%%%  

\addtolength{\marginparwidth}{-20pt} % robovi za tisk
\addtolength{\oddsidemargin}{40pt}
\addtolength{\evensidemargin}{-40pt}

\renewcommand{\baselinestretch}{1.3} % ustrezen razmik med vrsticami
\setlength{\headheight}{15pt}        % potreben prostor na vrhu
\renewcommand{\chaptermark}[1]%
{\markboth{\MakeUppercase{\thechapter.\ #1}}{}} \renewcommand{\sectionmark}[1]%
{\markright{\MakeUppercase{\thesection.\ #1}}} \renewcommand{\headrulewidth}{0.5pt} \renewcommand{\footrulewidth}{0pt}
\fancyhf{}
\fancyhead[LE,RO]{\sl \thepage} 
%\fancyhead[LO]{\sl \rightmark} \fancyhead[RE]{\sl \leftmark}
\fancyhead[RE]{\sc \tauthor}              % dodal Solina
\fancyhead[LO]{\sc Diplomska naloga}     % dodal Solina


\newcommand{\BibTeX}{{\sc Bib}\TeX}

%%%%%%%%%%%%%%%%%%%%%%%%%%%%%%%%%%%%%%%%
% naslovi
%%%%%%%%%%%%%%%%%%%%%%%%%%%%%%%%%%%%%%%%  


\newcommand{\autfont}{\Large}
\newcommand{\titfont}{\LARGE\bf}
\newcommand{\clearemptydoublepage}{\newpage{\pagestyle{empty}\cleardoublepage}}
\setcounter{tocdepth}{1}	      % globina kazala

%%%%%%%%%%%%%%%%%%%%%%%%%%%%%%%%%%%%%%%%
% konstrukti
%%%%%%%%%%%%%%%%%%%%%%%%%%%%%%%%%%%%%%%%  
\newtheorem{izrek}{Izrek}[chapter]
\newtheorem{trditev}{Trditev}[izrek]
\newenvironment{dokaz}{\emph{Dokaz.}\ }{\hspace{\fill}{$\Box$}}

%%%%%%%%%%%%%%%%%%%%%%%%%%%%%%%%%%%%%%%%%%%%%%%%%%%%%%%%%%%%%%%%%%%%%%%%%%%%%%%
%% PDF-A
%%%%%%%%%%%%%%%%%%%%%%%%%%%%%%%%%%%%%%%%%%%%%%%%%%%%%%%%%%%%%%%%%%%%%%%%%%%%%%%


%%%%%%%%%%%%%%%%%%%%%%%%%%%%%%%%%%%%%%%% 
% define medatata
%%%%%%%%%%%%%%%%%%%%%%%%%%%%%%%%%%%%%%%% 
\def\Title{\ttitle}
\def\Author{\tauthor, matjaz.kralj@fri.uni-lj.si}
\def\Subject{\ttitleEn}
\def\Keywords{\tkeywordsEn}

%%%%%%%%%%%%%%%%%%%%%%%%%%%%%%%%%%%%%%%% 
% \convertDate converts D:20080419103507+02'00' to 2008-04-19T10:35:07+02:00
%%%%%%%%%%%%%%%%%%%%%%%%%%%%%%%%%%%%%%%% 
\def\convertDate{%
    \getYear
}

{\catcode`\D=12
 \gdef\getYear D:#1#2#3#4{\edef\xYear{#1#2#3#4}\getMonth}
}
\def\getMonth#1#2{\edef\xMonth{#1#2}\getDay}
\def\getDay#1#2{\edef\xDay{#1#2}\getHour}
\def\getHour#1#2{\edef\xHour{#1#2}\getMin}
\def\getMin#1#2{\edef\xMin{#1#2}\getSec}
\def\getSec#1#2{\edef\xSec{#1#2}\getTZh}
\def\getTZh +#1#2{\edef\xTZh{#1#2}\getTZm}
\def\getTZm '#1#2'{%
    \edef\xTZm{#1#2}%
    \edef\convDate{\xYear-\xMonth-\xDay T\xHour:\xMin:\xSec+\xTZh:\xTZm}%
}

\expandafter\convertDate\pdfcreationdate 

%%%%%%%%%%%%%%%%%%%%%%%%%%%%%%%%%%%%%%%%
% get pdftex version string
%%%%%%%%%%%%%%%%%%%%%%%%%%%%%%%%%%%%%%%% 
\newcount\countA
\countA=\pdftexversion
\advance \countA by -100
\def\pdftexVersionStr{pdfTeX-1.\the\countA.\pdftexrevision}


%%%%%%%%%%%%%%%%%%%%%%%%%%%%%%%%%%%%%%%%
% XMP data
%%%%%%%%%%%%%%%%%%%%%%%%%%%%%%%%%%%%%%%%  
\usepackage{xmpincl}
\includexmp{pdfa-1b}

%%%%%%%%%%%%%%%%%%%%%%%%%%%%%%%%%%%%%%%%
% pdfInfo
%%%%%%%%%%%%%%%%%%%%%%%%%%%%%%%%%%%%%%%%  
\pdfinfo{%
    /Title    (\ttitle)
    /Author   (\tauthor, damjan@cvetan.si)
    /Subject  (\ttitleEn)
    /Keywords (\tkeywordsEn)
    /ModDate  (\pdfcreationdate)
    /Trapped  /False
}


%%%%%%%%%%%%%%%%%%%%%%%%%%%%%%%%%%%%%%%%%%%%%%%%%%%%%%%%%%%%%%%%%%%%%%%%%%%%%%%
%%%%%%%%%%%%%%%%%%%%%%%%%%%%%%%%%%%%%%%%%%%%%%%%%%%%%%%%%%%%%%%%%%%%%%%%%%%%%%%

\begin{document}
\selectlanguage{slovene}
\frontmatter
\setcounter{page}{1} %
\renewcommand{\thepage}{}       % preprecimo težave s številkami strani v kazalu
\newcommand{\sn}[1]{"`#1"'}                    % dodal Solina (slovenski narekovaji)

%%%%%%%%%%%%%%%%%%%%%%%%%%%%%%%%%%%%%%%%
%naslovnica
 \thispagestyle{empty}%
   \begin{center}
    {\large\sc Univerza v Ljubljani\\%
%      Fakulteta za elektrotehniko\\% za študijski program Multimedija
%      Fakulteta za upravo\\% za študijski program Upravna informatika
      Fakulteta za računalništvo in informatiko\\%
%      Fakulteta za matematiko in fiziko\\% za študijski program Računalništvo in matematika
     }
    \vskip 10em%
    {\autfont \tauthor\par}%
    {\titfont \ttitle \par}%
    {\vskip 3em \textsc{DIPLOMSKO DELO\\[5mm]         % dodal Solina za ostale študijske programe
%    VISOKOŠOLSKI STROKOVNI ŠTUDIJSKI PROGRAM\\ PRVE STOPNJE\\ RAČUNALNIŠTVO IN INFORMATIKA}\par}%
     UNIVERZITETNI  ŠTUDIJSKI PROGRAM\\ PRVE STOPNJE\\ RAČUNALNIŠTVO IN INFORMATIKA}\par}%
%    INTERDISCIPLINARNI UNIVERZITETNI\\ ŠTUDIJSKI PROGRAM PRVE STOPNJE\\ MULTIMEDIJA}\par}%
%    INTERDISCIPLINARNI UNIVERZITETNI\\ ŠTUDIJSKI PROGRAM PRVE STOPNJE\\ UPRAVNA INFORMATIKA}\par}%
%    INTERDISCIPLINARNI UNIVERZITETNI\\ ŠTUDIJSKI PROGRAM PRVE STOPNJE\\ RAČUNALNIŠTVO IN MATEMATIKA}\par}%
    \vfill\null%
% izberite pravi habilitacijski naziv mentorja!
    {\large \textsc{Mentor}: viš. pred./doc./izr. prof./prof. dr. Peter Klepec\par}%
   {\large \textsc{Somentor}:  viš. pred./doc./izr. prof./prof. dr.  Martin Krpan \par}%
    {\vskip 2em \large Ljubljana, 2019 \par}%
\end{center}
% prazna stran
%\clearemptydoublepage      % dodal Solina (izjava o licencah itd. se izpiše na hrbtni strani naslovnice)

%%%%%%%%%%%%%%%%%%%%%%%%%%%%%%%%%%%%%%%%
%copyright stran
\thispagestyle{empty}
\vspace*{8cm}

\noindent
{\sc Copyright}. 
Rezultati diplomske naloge so intelektualna lastnina avtorja in matične fakultete Univerze v Ljubljani.
Za objavo in koriščenje rezultatov diplomske naloge je potrebno pisno privoljenje avtorja, fakultete ter mentorja.

\begin{center}
\mbox{}\vfill
\emph{Besedilo je oblikovano z urejevalnikom besedil \LaTeX.}
\end{center}
% prazna stran
\clearemptydoublepage

%%%%%%%%%%%%%%%%%%%%%%%%%%%%%%%%%%%%%%%%
% stran 3 med uvodnimi listi
\thispagestyle{empty}
\
\vfill

\bigskip
\noindent\textbf{Kandidat:} Ime Priimek\\
\noindent\textbf{Naslov:} Naslov diplome v slovenščini\\
% vstavite ustrezen naziv študijskega programa!
\noindent\textbf{Vrsta naloge:} npr. Diplomska naloga na univerzitetnem programu prve stopnje Računalništvo in informatika \\
% izberite pravi habilitacijski naziv mentorja!
\noindent\textbf{Mentor:} viš. pred. / doc. / izr. prof. / prof. dr. Ime Priimek\\
\noindent\textbf{Somentor:} isto kot za mentorja

\bigskip
\noindent\textbf{Opis:}\\
Besedilo teme diplomskega dela študent prepiše iz študijskega informacijskega sistema, kamor ga je vnesel mentor. 
V nekaj stavkih bo opisal, kaj pričakuje od kandidatovega diplomskega dela. 
Kaj so cilji, kakšne metode naj uporabi, morda bo zapisal tudi ključno literaturo.

\bigskip
\noindent\textbf{Title:} Naslov diplome v angleščini

\bigskip
\noindent\textbf{Description:}\\
opis diplome v angleščini

\vfill



\vspace{2cm}

% prazna stran
\clearemptydoublepage

% zahvala
\thispagestyle{empty}\mbox{}\vfill\null\it%
\noindent
Na tem mestu zapišite, komu se zahvaljujete za izdelavo diplomske naloge. Pazite, da ne boste koga pozabili. Utegnil vam bo zameriti. Temu se da izogniti tako, da celotno zahvalo izpustite.
\rm\normalfont

% prazna stran
\clearemptydoublepage

%%%%%%%%%%%%%%%%%%%%%%%%%%%%%%%%%%%%%%%%
% posvetilo, če sama zahvala ne zadošča :-)
\thispagestyle{empty}\mbox{}{\vskip0.20\textheight}\mbox{}\hfill\begin{minipage}{0.55\textwidth}%
Svoji dragi Alenčici.
\normalfont\end{minipage}

% prazna stran
\clearemptydoublepage


%%%%%%%%%%%%%%%%%%%%%%%%%%%%%%%%%%%%%%%%
% kazalo
\pagestyle{empty}
\def\thepage{}% preprecimo tezave s stevilkami strani v kazalu
\tableofcontents{}


% prazna stran
\clearemptydoublepage

%%%%%%%%%%%%%%%%%%%%%%%%%%%%%%%%%%%%%%%%
% seznam kratic

\chapter*{Seznam uporabljenih kratic}  % spremenil Solina, da predolge vrstice ne gredo preko desnega roba

\begin{comment}
\begin{tabular}{l|l|l}
  {\bf kratica} & {\bf angleško} & {\bf slovensko} \\ \hline
  % after \\: \hline or \cline{col1-col2} \cline{col3-col4} ...
  {\bf CA} & classification accuracy & klasifikacijska točnost \\
  {\bf DBMS} & database management system & sistem za upravljanje podatkovnih baz \\
  {\bf SVM} & support vector machine & metoda podpornih vektorjev \\
  \dots & \dots & \dots \\
\end{tabular}
\end{comment}

\noindent\begin{tabular}{p{0.1\textwidth}|p{.4\textwidth}|p{.4\textwidth}}    % po potrebi razširi prvo kolono tabele na račun drugih dveh!
  {\bf kratica} & {\bf angleško}                             & {\bf slovensko} \\ \hline
  {\bf CA}      & classification accuracy               & klasifikacijska točnost \\
  {\bf DBMS} & database management system & sistem za upravljanje podatkovnih baz \\
  {\bf SVM}   & support vector machine              & metoda podpornih vektorjev \\
%  \dots & \dots & \dots \\
\end{tabular}


% prazna stran
\clearemptydoublepage

%%%%%%%%%%%%%%%%%%%%%%%%%%%%%%%%%%%%%%%%
% povzetek
\addcontentsline{toc}{chapter}{Povzetek}
\chapter*{Povzetek}

\noindent\textbf{Naslov:} \ttitle
\bigskip

\noindent\textbf{Avtor:} \tauthor
\bigskip

%\noindent\textbf{Povzetek:} 
\noindent V vzorcu je predstavljen postopek priprave diplomskega dela z uporabo okolja \LaTeX. Vaš povzetek mora sicer vsebovati približno 100 besed, ta tukaj je odločno prekratek.
Dober povzetek vključuje: (1) kratek opis obravnavanega problema, (2) kratek opis vašega pristopa za reševanje tega problema in (3) (najbolj uspešen) rezultat ali prispevek magistrske naloge.

\bigskip

\noindent\textbf{Ključne besede:} \tkeywords.
% prazna stran
\clearemptydoublepage

%%%%%%%%%%%%%%%%%%%%%%%%%%%%%%%%%%%%%%%%
% abstract
\selectlanguage{english}
\addcontentsline{toc}{chapter}{Abstract}
\chapter*{Abstract}

\noindent\textbf{Title:} \ttitleEn
\bigskip

\noindent\textbf{Author:} \tauthor
\bigskip

%\noindent\textbf{Abstract:} 
\noindent This sample document presents an approach to typesetting your BSc thesis using \LaTeX. 
A proper abstract should contain around 100 words which makes this one way too short.
\bigskip

\noindent\textbf{Keywords:} \tkeywordsEn.
\selectlanguage{slovene}
% prazna stran
\clearemptydoublepage

%%%%%%%%%%%%%%%%%%%%%%%%%%%%%%%%%%%%%%%%
\mainmatter
\setcounter{page}{1}
\pagestyle{fancy}

\chapter{Uvod}
Prvi koristen nasvet v zvezi uporabo \LaTeX{a} je, da v celoti preberete ta dokument!

Datoteka {\tt vzorec\_dip\_Seminar.tex} na kratko opisuje, kako se pisanja diplomskega dela lotimo z uporabo programskega okolja \LaTeX~\cite{lamport,nenajkrajsi}. 
V tem dokumentu bomo predstavili nekaj njegovih prednosti in hib. 
Kar se slednjih tiče, nam pride na misel ena sama. 
Ko se srečamo z njim prvič, nam izgleda morda kot kislo jabolko, nismo prepričani, ali bi želeli vanj ugrizniti. 
Toda prav iz kislih jabolk lahko pripravimo odličen jabolčni zavitek in s praktičnim preizkusom \LaTeX a najlažje pridemo na njegov pravi okus.

\LaTeX\ omogoča logično urejanje besedil, ki ima v primerjavi z vizualnim urejanjem številne prednosti, saj se problema urejanja besedil loti s programerskega stališča.
Logično urejanje besedil omogoča večjo konsistentnost, uniformnost in  prenosljivost besedil. 
Vsebinska struktura nekega besedila pa se odraža v strukturiranem \LaTeX ovem kodiranju besedila.

V \ref{ch0}.~poglavju bomo spoznali osnovne gradnike \LaTeX{a}.
V \ref{ch1}.~poglavju bomo na hitro spoznali besedilne konstrukte kot so izreki, enačbe in dokazi. 
Naučili se bomo, kako se na njih sklicujemo. 
\ref{ch2}.~poglavje bo predstavilo vključevanje plovk: slik in tabel. 
Poglavje \ref{stroka} na kratko predstavi tipične sestavne dele strokovnega besedila.
V \ref{slo}.~poglavju omenjamo nekaj najpogostejših slovničnih napak, ki jih delamo v slovenščini.
V \ref{latex}.~poglavju je še nekaj koristnih praktičnih nasvetov v zvezi z uporabo \LaTeX{a}.
V \ref{lit}.~poglavju se bomo srečali s sklicevanjem na literaturo,
\ref{PDF}.~poglavje pa govori o formatu PDF/A, v katerem morate svojo diplomo oddati v sistemu STUDIS.
Sledil bo samo še zaključek.

Ta vzorec ni priročnik za uporabo \LaTeX{a}, saj razloži le nekatere osnovne ukaze, druge funkcionalnosti pa le omeni. Kako se jih uporablja
pa naj bralec poišče drugje.


\chapter{Osnovni gradniki \LaTeX{a}}
\label{ch0}

\LaTeX\ bi lahko najbolj preprosto opisali kot programski jezik namenjen oblikovanju besedil.
Tako kot vsak visokonivojski programski jezik ima tudi \LaTeX\  številne ukaze za oblikovanje  besedila in okolja, ki omogočajo strukturiranje besedila.

Vsi \LaTeX ovi ukazi se začnejo z levo poševnico  \verb=\=, okolja pa definiramo bodisi s parom zavitih oklepajev \{ in \} ali z ukazoma \verb=\begin{ }= in   \verb=\end{ }=.
Ukazi imajo lahko tudi argumente, obvezni argumenti so podani v zavitih oklepajih, opcijski argumenti pa v oglatih oklepajih.

Z ukazi torej definiramo naslov in imena avtorjev besedila, poglavja in podpoglavja in po potrebi bolj podrobno strukturiramo besedila na spiske, navedke itd.
Posebna okolja so namenjena zapisu matematičnih izrazov, kratki primeri so v naslednjem poglavju.

Vse besedilne konstrukte lahko poimenujemo in se s pomočjo teh imen nato  kjerkoli v besedilu na njih  tudi sklicujemo.

\LaTeX\ sam razporeja besede v odstavke tako, da optimizira razmike med besedami v celotnem odstavku.
Nov odstavek začnemo tako, da izpustimo v izvornem besedilu prazno vrstico. Da besedilo skoči v novo vrstico pa ukažemo z dvema levima poševnicama.
Število presledkov med besedami v izvornem besedilo ni pomembno.


\chapter{Matematično okolje in sklicevanje na besedilne konstrukte}
\label{ch1}



Matematična ali popolna indukcija je eno prvih orodij, ki jih spoznamo za dokazovanje trditev pri matematičnih predmetih.
\begin{izrek}
\label{iz:1}
Za vsako naravno število $n$ velja
\begin{equation}
n < 2^n.
\label{eq:1}
\end{equation}
\end{izrek}
\begin{dokaz}
Dokazovanje z indukcijo zahteva, da neenakost~\eqref{eq:1} najprej preverimo za najmanjše naravno število -- $0$. 
Res, ker je $0 < 1 = 2^0$, je neenačba~\eqref{eq:1} za $n=0$ izpolnjena.

Sledi indukcijski korak. S predpostavko, da je neenakost~\eqref{eq:1} veljavna pri nekem naravnem številu $n$, je potrebno pokazati, da je ista neenakost v veljavi tudi pri njegovem nasledniku -- naravnem številu $n+1$. 
Računajmo.
\begin{align}
n+1 & < 2^n + 1       \label{eq:2}\\
       & \le 2^n + 2^n \label{eq:3}\\
       & = 2^{n+1}       \nonumber
\end{align}
Neenakost~\eqref{eq:2} je posledica indukcijske predpostavke, neenakost~\eqref{eq:3} pa enostavno dejstvo, da je za vsako naravno število $n$ izraz $2^n$ vsaj tako velik kot 1. 
S tem je dokaz Izreka~\ref{iz:1} zaključen.
\end{dokaz}

Opazimo, da je \LaTeX\ številko izreka podredil številki poglavja.
Na podoben način se lahko sklicujemo tudi na druge besedilne konstrukte, kot so med drugim poglavja, podpoglavja in plovke, ki jih bomo spoznali v naslednjem poglavju.


\chapter{Plovke: slike in tabele}
\label{ch2}

Slike in daljše tabele praviloma vključujemo v dokument kot plovke. 
Pozicija plovke v končnem izdelku ni pogojena s tekom besedila, temveč z izgledom strani. 
\LaTeX\ bo skušal plovko postaviti samostojno, praviloma na mestu, kjer se pojavi v izvornem besedilu, sicer pa na 
vrhu strani, na kateri se na takšno plovko prvič sklicujemo. 
Pri tem pa bo na vsako stran končnega izdelka želel postaviti tudi sorazmerno velik del besedila. 
V skrajnem primeru, če imamo res preveč plovk na enem mestu besedila, ali če je plovka previsoka, se bo \LaTeX\ odločil za stran popolnoma zapolnjeno s plovkami.

Poleg tega, da na položaj plovke vplivamo s tem, kam jo umestimo v izvorno besedilo, lahko na položaj plovke na posamezni strani prevedenega besedila dodatno vplivamo z opcijami \texttt{here, top} in \texttt{bottom}.
Zelo velike slike je najbolje postaviti na posebno stran z opcijo \texttt{page}.
Skaliranje slik po njihovi širini lahko prilagodimo širini strani tako, da kot enoto za dolžino uporabimo kar širino strani, npr. \verb=0.5\textwidth= bo raztegnilo sliko na polovico širine strani.
Slike lahko po potrebi tudi zavrtimo za 90 stopinj, tako da bodo podrobnosti na sliki lažje berljive in da bo prostor na papirju bolje izkoriščen.

Na vse plovke se moramo v besedilu sklicevati, saj kot beseda pove, plovke plujejo po besedilu in se ne pojavijo točno tam, kjer nastopajo v izvornem besedilu.
Sklic na plovko v besedilu in sama plovka naj bosta čimbližje skupaj, tako da bralcu ne bo potrebno listati po diplomi. 
Upoštevajte pa, da se naloge tiska dvostransko in da se hkrati vidi dve strani v dokumentu!
Na to, kje se bo slika ali druga plovka pojavila v postavljenem besedilu torej najbolj vplivamo tako, da v izvorni kodi plovko premikamo po besedilu nazaj ali naprej!

Tabele ja najbolje oblikovati kar neposredno v \LaTeX u, saj za oblikovanje tabel obstaja zelo fleksibilno okolje \texttt{tabular} (glej tabelo~\ref{tbl:1}).
Slike pa je po drugi strani  pogosto najla\v zje oblikovati oziroma izdelati z drugimi orodji in programi, rezultate shraniti v formatu {\tt .pdf}  in se v \LaTeX u le sklicevati na ustrezno slikovno datoteko.

Knjižnica \url{https://en.wikibooks.org/wiki/LaTeX/PGF/TikZ} 
pa omo\-go\-ča risanje raznovrstnih grafov neposredno v okolju \LaTeX .


\section{Formati slik}

Bitne slike, vektorske slike, kakršnekoli slike, z \LaTeX{}om lahko vključimo vse.
Slika~\ref{pic1} je v formatu {\tt .pdf}.
\begin{figure}[h]
\begin{center}
\includegraphics[width=0.6\textwidth]{pic1.pdf}
\end{center}
\caption{Herschelov graf, vektorska grafika.}
\label{pic1}
\end{figure}
Pa res lahko vključimo slike katerihkoli formatov? 
Žal ne. 
Programski paket \LaTeX\ lahko uporabljamo v več dialektih. 
Ukaz {\tt latex} ne mara vključenih slik v formatu Portable Document Format {\tt .pdf}, ukaz {\tt pdflatex} pa ne prebavi slik v Encapsulated Postscript Formatu {\tt .eps}.
Strnjeno je vključevanje različnih vrst slikovnih datotek prikazano v tabeli~\ref{tbl:1}.

\begin{table}
\begin{center}
\begin{tabular}{l|ccc}
ukaz/format & {\tt .pdf} & {\tt .eps} & ostali formati \\ \hline
{\tt pdflatex} & da & ne & da \\
{\tt latex}   & ne & da  & da
\end{tabular}
\end{center}
\caption{Kompatibilnost različnih formatov slikovnih datotek z različnimi dialekti  \LaTeX a.}
\label{tbl:1}
\end{table}

Nasvet? 
Odločite se za uporabo ukaza {\tt pdflatex}. Vaš izdelek bo brez vmesnih stopenj na voljo v {.pdf} formatu in ga lahko odnesete v vsako tiskarno. 
Če morate na vsak način vključiti sliko, ki jo imate v {\tt .eps} formatu, jo vnaprej pretvorite v alternativni format, denimo {\tt .pdf}.

Včasih se da v okolju za uporabo programskega paketa \LaTeX\ nastaviti na kakšen način bomo prebavljali vhodne dokumente. 
Spustni meni na Sliki~\ref{pic2} odkriva uporabo \LaTeX{}a v njegovi pdf inkarnaciji --- {\tt pdflatex}.
\begin{figure}[t]
\begin{center}
\includegraphics[width=10cm]{pic2.png}
\end{center}
\caption{Kateri dialekt uporabljati?}
\label{pic2}
\end{figure}
Vključena slika~\ref{pic2} je seveda bitna.

Na vse tabele se moramo v besedilu, podobno kot na slike, tudi sklicevati, saj kot plovke v oblikovanem besedilo niso nujno na istem mestu kot v izvornem besedilu.


\subsection{Podnapisi k slikam in tabelam}

Vsaki sliki ali tabeli moramo dodati podnapis, ki na kratko pojasnuje, kaj je na sliki ali tabeli. 
Če nekdo le prelista diplomsko delo, naj bi že iz slik in njihovih podnapisov lahko na grobo razbral, kakšno temo naloga obravnava.

Če slike povzamemo iz drugih virov, potem se moramo v podnapisu k taki sliki sklicevati na ta vir!


\chapter{Struktura strokovnih besedil}
\label{stroka}

Strokovna besedila imajo ustaljeno strukturo, da bi lahko hitreje in lažje brali in predvsem razumeli taka besedila, saj načeloma vemo vnaprej, 
kje v besedilu se naj bi nahajale določene informacije.

Najbolj osnovna struktura strokovnega besedila je:
\begin{description}
\item[naslov besedila,] ki naj bo sicer kratek, a kljub temu dovolj poveden o vsebini besedila,
\item[imena avtorjev] so običajno navedena po teži prispevka, prvi avtor je tisti, ki je besedilo dejansko pisal, zadnji pa tisti, ki je raziskavo vodil,
\item[kontaktni podatki] -- poleg imena in naslova institucije je potreben vsaj naslov elektronske pošte,
\item[povzetek] je kratko besedilo, ki povsem samostojno povzame vsebino in izpostavi predvsem  glavne rezultate ali zaključke,
\item[ključne besede] so tudi namenjene iskanju vsebin med množico člankov,
\item[uvodno poglavje] uvede bralca v tematiko besedila, razloži kaj je namen besedila, predstavi področje o katerem besedilo piše 
(če temu ni namenjeno v celoti posebno poglavje) ter na kratko predstavi strukturo celotnega besedila,
\item[poglavja] tvorijo zaokrožene celote, ki se po potrebi še nadalje členijo na podpoglavja, namenjena so recimo opisu orodij, 
ki smo jih uporabili pri delu, teoretičnim rezultatom ali predstavitvi rezultatov, ki smo jih dosegli,
\item[zaključek] še enkrat izpostavi glavne rezultate ali ugotovitve, jih primerja z dosedanjimi in morebiti poda tudi ideje za nadaljne delo,
\item[literatura] je seznam vseh virov, na katere smo se pri svojem delu opirali, oziroma smo se na njih sklicevali v svojem besedilu.
\end{description}

Naslove poglavij in podpoglavij izbiramo tako, da lahko bralec že pri prelistavanju diplome in branju naslovov
v grobem ugotovi, kaj je vsebina diplomskega dela.

Strokovna besedila običajno pišemo v prvi osebi množine, v nevtralnem in umirjenem tonu. 
Uporaba sopomenk ni zaželjena, saj želimo zaradi lažjega razumevanja za iste pojme vseskozi uporabljati iste besede.
Najpomenbnejše ugotovitve je smiselno večkrat zapisati, na primer v povzetku, uvodu, glavnem delu in zaključku.
Vse trditve naj bi temeljile bodisi na lastnih ugotovitvah (izpeljavah, preizkusih, testiranjih) ali pa z navajanjem ustreznih virov.

Največ se lahko naučimo s skrbnim branjem dobrih zgledov takih besedil.


\chapter{Pogoste napake pri pisanju v slovenščini}  % poglavje dodal Solina
\label{slo}

V slovenščini moramo paziti  pri uporabi pridevnikov, ki se ne sklanjajo kot so npr. kratice. 
Pravilno pišemo model CAD in \textbf{ne} CAD model!

Pri sklanjanju tujih imen ne uporabljamo vezajev, pravilno je Applov operacijski sistem in \textbf{ne} Apple-ov.

Pika, klicaj in vprašaj so levostični: pred njimi ni presledka, za njimi pa. 
Klicajev in vprašajev se v strokovnih besedilih načeloma izogibamo. Oklepaji so desnostični in zaklepaji levostični (takole).

V slovenščini pišemo narekovaje drugače kot v angleščini!   
Običajno uporabljamo dvojne spodnje-zgornje narekovaje:  \sn{slovenski narekovaji}.
Za slovenske narekovaje je v tej LaTeXovi predlogi definiran nov ukaz \verb+ \sn{ ... }+.

Vezaj  je levo in desno stičen: \verb=slovensko-angleški slovar= in ga pišemo z enim pomišljajem.

V slovenščini je pred in po pomišljaju presledek, ki ga v LaTeXu pišemo z dvema pomišljajema: \verb=Pozor -- hud pes!=
V angleščini pa je za razliko pomišljaj levo in desno stičen in se v LaTeXu piše s tremi  pomišljaji: \verb=---=.
S stičnim pomišljajem pa lahko nadomeščamo predlog od \dots do, denimo pri navajanju strani, npr. preberite strani 7--11 (\verb=7--11=).



\sn{Pred ki, ko, ker, da, če vejica skače}. To osnovnošolsko pravilo smo v življenju po potrebi uporabljali, dopolnili, morda celo pozabili. 
Pravilo sicer drži, ampak samo če je izpolnjenih kar nekaj pogojev (npr. da so ti vezniki samostojni, enobesedni, ne gre za vrivek itd.).
Povedki so med seboj ločeni z vejicami, razen če so zvezani z in, pa, ter, ne–ne, niti–niti, ali, bodisi, oziroma.
Sicer pa je bolje pisati kratke stavke kot pretirano dolge.

V računalništvu se stalno pojavljajo novi pojmi in nove besede, za katere pogosto še ne obstajajo uveljavljeni slovenski izrazi.
Kadar smo v dvomih, kateri slovenski izraz je primeren, si lahko pomagamo z Računalniškim slovarčkom~\cite{slovarcek}.





\chapter{Koristni nasveti pri pisanju v \LaTeX{u}}   % poglavje dodal Solina
\label{latex}

Programski paket \LaTeX\ je bil prvotno predstavljen v priročniku~\cite{lamport} in je v resnici nadgradnja sistema \TeX\ avtorja Donalda Knutha~\cite{knuth}, 
znanega po svojih knjigah o umetnosti programiranja, 
ter Knuth-Bendixovem algoritmu~\cite{knuth1983simple}.

Različnih implementacij \LaTeX{}a je cela vrsta.
Za OS X priporočamo TeXShop, za Windows PC pa MikTeX.
Spletna verzija, ki poenostavi sodelovanje pri pisanju, je Overleaf.

Včasih smo si pri pisanju v \LaTeX{}u  pomagali predvsem s tiskanimi pri\-ro\-čni\-ki, danes pa je enostavneje in hitreje, da ob vsakem problemu za pomoč enostavno povprašamo Google, 
saj je na spletu cela vrsta forumov za pomoč pri \TeX{}iranju.

\LaTeX\ včasih ne zna deliti slovenskih besed, ki vsebujejo črke s strešicami. 
Če taka beseda štrli preko desnega roba,  \LaTeX{}u pokažemo, kje lahko tako besedo deli, takole: \verb=ra\-ču\-nal\-ni\-štvo=.
Katere vrstice štrlijo preko desnega roba, se lahko prepričamo tako, da dokument prevedemo s vključeno opcijo \texttt{draft}: \verb=\documentclass[a4paper, 12pt, draft]{book}=.


Predlagamo, da v izvornem besedilu začenjate vsak stavek v novi vrstici, saj \LaTeX\ sam razporeja besede po vrsticah postavljenega besedila. 
Bo pa zato iskanje po izvornem besedilu in popravljanje veliko hitrejše. 
Večina sistemov za \TeX{}iranje sicer omogoča s klikanjem enostavno prestopanje  iz prevedenega besedila na ustrezno mesto v izvornem besedilu in obratno.

Boljšo preglednost dosežemo, tako kot pri pisanju programske kode, tudi z izpuščanjem praznih vrstic za boljšo preglednost strukture izvornega besedila.

S pomočjo  okolja \verb=\begin{comment} ... \end{comment}= lahko  hkrati zakomentiramo več vrstic izvornega besedila.

Pri spreminjanju in dodajanju izvornega besedila je najbolje pogosto prevajati, da se sproti prepričamo, če so naši nameni izpolnjeni pravilno.

Kadar besedilo, ki je že bilo napisano z nekim vizualnim urejevalnikom (npr. z Wordom), želimo prenesti v \LaTeX, je tudi najbolje to delati postopoma s posameznimi bloki besedila, 
tako da lahko morebitne napake hitro identificiramo in odpravimo.
Za prevajanje Wordovih datotek v \LaTeX\ sicer obstajajo prevajalniki, ki pa običajno ne generirajo tako čisto logično strukturo besedila, kot jo  \LaTeX\ omogoča.
Hiter in enostaven način prevedbe besedila, ki  zahteva sicer ročne dopolnitve, poteka tako, da besedilo urejeno z vizualnim urejevalnikom najprej shranimo v formatu pdf, 
nato pa to besedilo uvozimo v urejevalnik, kjer urejamo izvorno besedilo v formatu \LaTeX.




\section{Pisave v \LaTeX u}

V  \LaTeX ovem okolju lahko načeloma uporabljamo poljubne pisave. 
Izbira poljubne pisave pa ni tako enostavna kot v vizualnih urejevalnikih besedil.
Posamezne oblikovno medseboj usklajene pisave so običajno združene v družine pisav.
V \LaTeX u se privzeta družina pisav imenuje Computer Modern,
kjer so poleg navadnih črk (roman v \LaTeX u) na voljo tudi kurzivne črke (\textit{italic} v \LaTeX u), 
krepke (\textbf{bold} v \LaTeX u), kapitelke (\textsc{small caps} v \LaTeX u), linearne črke ({\textsf{san serif} v \LaTeX u)                                                                                                                                                                                                                                                                                                                         
in druge pisave.
V istem dokumentu zaradi skladnega izleda uporabljamo običajno le pisave ene družine. 

Ko začenjamo uporabljati \LaTeX, je zato najbolj smiselno uporabljati kar privzete pisave, s katerimi je napisan tudi ta dokument.
Z ustreznimi ukazi  lahko nato preklapljamo med navadnimi, kurzivnimi, krepkimi in drugimi pisavami. 
Zelo enostavna je tudi izbira velikosti črk.
\LaTeX\  odlično podpira večjezičnost, tudi v sklopu istega dokumenta, saj obstajajo pisave za praktično vse jezike, tudi take, ki ne uporabljajo latinskih črk.

Za prikaz programske kode se pogosto uporablja pisava, kjer imajo vse črke enako širino, kot so  črke na mehanskem pisalnem stroju ({\texttt{typewriter} v \LaTeX u).

Najbolj priročno okolje za pisanje kratkih izsekov programske kode je okolje \texttt{verbatim}, saj ta ohranja vizualno organizacijo izvornega besedila in ima privzeto pisavo pisalnega stroja.

\begin{verbatim}
for (i = 0; i < 100; i++)
   for (j = i; j < 10; j++)
      some_function(i, j);
\end{verbatim}


\chapter{Kaj pa literatura?}
\label{lit}

Kot smo omenili že v uvodu, je pravi način za citiranje literature uporaba \BibTeX{}a~\cite{bib}. 
\BibTeX\ zagotovi, da nobene obvezne informacije pri določeni vrsti literature ne izpustimo in da vse informacije o določeni vrsti vira dosledno navajamo na enak način.

Osnovna ideja \BibTeX{a} je, da vse informacije o literaturi zapisujemo v posebno datoteko, v našem primeru je to \texttt{literatura.bib}.
Vsakemu viru v tej datoteki določimo simbolično ime.
V  našem primeru je v tej datoteki nekaj najbolj značilnih zvrsti literature, kot so knjige~\cite{lamport}, 
članki v revijah~\cite{leonardo} in zbornikih konferenc~\cite{poglavje_springer}, 
spletni viri~\cite{bib,slovarcek,video}, 
tehnično poročilo~\cite{andersen2012kinect}, 
diplome~\cite{diploma} itd.
Diploma~\cite{diploma} iz leta 1990 je bila prva diploma na Fakulteti za elektrotehniko in računalništvo, ki je bila oblikovana z \LaTeX om!
Novej\v se reference, ki so spletnih straneh svojih založnikov arhivirane v elektronski obliki, imajo določeno \v stevilko DOI: \url{http://dx.doi.org}, ki jo tudi lahko vključimo v izpis literature in omogoča neposredno povezavo do te reference~\cite{Kljun2018}.

Po vsaki spremembi pri sklicu na literaturo moramo najprej prevesti izvorno besedilo s prevajalnikom \LaTeX, nato s prevajalnikom  \BibTeX, ki ustvari datoteko  {\tt vzorec\_dip\_Seminar.bbl}, in nato še dvakrat s prevajalnikom  \LaTeX.

Kako natančno se spisek literature nato izpiše (ali po vrstnem redu sklicevanja, ali po abecedi priimkov prvih avtorjev, ali se imena avtorjev pišejo pred priimki itd.) je odvisno od stilske datoteke.
V diplomi bomo uporabili osnovno stilsko datoteko \texttt{plain}, oz. \texttt{plainnat}, ki vire razporedi po abecedi.
Zato je potrebno pri določenih zvrsteh literature, ki nima avtorjev, dodati polje \texttt{key}, ki določi vrstni red vira po abecedi.

Z uporabo \BibTeX{a} v slovenščini je še nekaj nedoslednosti, saj so pomožne besede, ki jih \BibTeX\ sam doda,  kot so \textit{editor},  \textit{pages} in besedica  \textit{and} pred zadnjim avtorjem, 
če ima vir več avtorjev~\cite{andersen2012kinect}, zapisane v angleščini,
čeprav smo izbrali opcijo \texttt{slovene} pri paketu \texttt{babel}.
To nedoslednost je možno popraviti z ročnim urejanjem datoteke {\tt vzorec\_dip\_Seminar.bbl}, 
kar pa je smiselno šele potem, ko bibliografije v datoteki \texttt{literatura.bib} ne bomo več spreminjali,
oziroma ne bomo več dodajali novih sklicev na literaturo v izvornem besedilu.
Vsebino datoteke {\tt vzorec\_dip\_Seminar.bbl} lahko na koncu urejanja tudi vključimo kar v izvorno besedilo diplome, tako da je vso besedilo, vključno z literaturo, zajeto le v eni datoteki.

Ko začenjamo uporabljati \BibTeX\ je lažje, če za urejanje datoteke .bib uporabljamo kar isti urejevalnik kot za urejanje datotek .tex, 
čeprav obstajajo tudi posebni urejevalniki oziroma programi za delo z \BibTeX om.

Le če se bomo na določen vir v besedilu tudi sklicevali, se bo pojavil tudi v spisku literature.
Tako je avtomatično zagotovljeno, da se na vsak vir v seznamu literature tudi sklicujemo v besedilu diplome.
V datoteki \texttt{.bib} imamo sicer lahko veliko več virov za literaturo, kot jih bomo uporabili v diplomi.

Vire v formatu \BibTeX\ lahko enostavno poiščemo in prekopiramo iz spletnih strani založnikov ali različnih akademskih spletnih portalov za iskanje znanstvene literature v našo datoteko \texttt{.bib}.
Izvoz  referenc v Google učenjaku še dodatno poenostavimo, če v nastavitvah izberemo \BibTeX\ kot želeni format za izvoz navedb.
Navedbe, ki jih na tak način prekopiramo, pa moramo pred uporabo vseeno preveriti, saj so taki navedki pogosto generirani povsem avtomatično in lahko vsebujejo napačne ali nepopolne podatke.

Pri sklicevanju na literaturo na koncu stavka moramo paziti, da je pika po ukazu \verb=\cite{ }=.
Da \LaTeX\ ne bi delil vrstico ravno tako, da bi sklic na literaturo v oglatih oklepajih začel novo vrstico, lahko pred sklicem na literaturo dodamo nedeljiv presledek: \verb=~\cite{ }=.


\section{Izbiranje virov za spisek literature}

Dandanes  se skoraj  vsi pri iskanju informacij vedno najprej lotimo iskanja preko svetovnega spleta.
Rezultati takega iskanja pa so pogosto spletne strani, ki danes obstajajo, jutri pa jih morda ne bo več, ali pa vsaj ne v taki obliki, kot smo jo prebrali.
Smisel navajanja literature pa je, da tudi po dolgih letih nekdo, ki bo bral vašo diplomo, lahko poišče vire, ki jih navajate v diplomi.
Taki viri pa so predvsem članki v znanstvenih revijah, ki se arhivirajo v knjižnicah, založniki teh revij pa večinoma omogočajo tudi elektronski dostop do arhiva vseh njihovih člankov.

Znanstveni rezultati, ki so objavljeni v obliki recenziranih člankov, bodisi v konferenčnih zbornikih, še bolje pa v znanstvenih revijah, so veliko bolj izčiščen in zanesljiv vir informacij, saj
so taki članki šli skozi recenzijski postopek.
Zato na svetovnem spletu začenjamo iskati vire za strokovna besedila predvsem preko akademskih spletnih portalov, kot so npr. Google učenjak, Research Gate ali Academia, saj
so na teh portalih rezultati iskanja le akademske publikacije.
Če je za dostop do nekega članka potrebno plačati, se obrnemo za pomoč in dodatne informacije na  našo knjižnico.

Če res ne gre drugače, pa je pomembno, da pri sklicevanju na spletni vir, vedno navedemo tudi datum, kdaj smo dostopali do tega vira.



\chapter{Sistem STUDIS in PDF/A}
\label{PDF}

Elektronsko verzijo diplome moramo oddati preko sistema STUDIS v formatu PDF/A ~\cite{pdfa}.
Natančneje v formatu PDF/A-1b. 

\LaTeX\ in omenjeni format imata še nekaj težav s sobivanjem. 
Paket \texttt{pdfx.sty}, ki naj bi \LaTeX{u} omogočal podporo formatu PDF/A ne deluje v skladu s pričakovanji. 
Ta predloga delno ustreza formatu, vsekakor dovolj, da jo študentski informacijski sistem sprejme. 
Znaten del rešitve je prispeval Damjan Cvetan.

V predlogi, poleg izvornega  dokumenta \texttt{.tex} in vloženih slik \texttt{pic1.pdf} in \texttt{pic2.png}, 
potrebujemo še predlogo datoteke z metapodatki \texttt{pdfa-1b.xmp} in datoteko z barvnim profilom \texttt{sRGBIEC1966-2.1.icm}.




\chapter{Sklepne ugotovitve}

Uporaba \LaTeX{a} in \BibTeX{a} je v okviru Diplomskega seminarja \textbf{obvezna}!
Izbira \LaTeX\ ali ne \LaTeX\ pri pisanju dejanske diplomske naloge pa je pre\-pu\-šče\-na dogovoru med vami in vašim mentorjem.

Res je, da so prvi koraki v \LaTeX{}u težavni. 
Ta dokument naj vam služi kot začetna opora pri hoji.
Pri kakršnihkoli nadaljnih vprašanjih ali napakah pa svetujem uporabo Googla, saj je spletnih strani za pomoč pri odpravljanju težav pri uporabi \LaTeX{}a ogromno.

Preden diplomo oddate na sistemu STUDIS, še enkrat preverite, če so slovenske besede, ki vsebujejo črke s strešicami,  pravilno deljene.
Poravnavo po vrsticah pa kontrolirajte tako, da izvorno datoteko prevedete z opcijo \texttt{draft}, kar vam pokaže predolge vrstice.


\cleardoublepage
\addcontentsline{toc}{chapter}{Literatura}
\bibliography{literatura}
\bibliographystyle{plainnat}

\end{document}

